\def\year{2016}\relax
%File: formatting-instruction.tex
\documentclass[letterpaper]{article}
\usepackage{aaai16}
\usepackage{times}
\usepackage{helvet}
\usepackage{courier}
\frenchspacing \setlength{\pdfpagewidth}{8.5in}
\setlength{\pdfpageheight}{11in}

%\pdfinfo{Title (Data-and-Model Driven Classical Planning)/Author (Roni Stern, Brendan Juba)}


\setcounter{secnumdepth}{0}



\usepackage{latexsym}
\usepackage{graphicx}
\usepackage{color}
\usepackage{multirow}
\usepackage{amsmath}
\usepackage{amssymb}
\usepackage{psfig}
\usepackage{subcaption}
\usepackage{tabularx}

\usepackage[ruled,vlined,linesnumbered]{algorithm2e}
\newtheorem{definition}{Definition}

%\nocopyright

\newcommand{\comment}[1]{}

\newcommand{\OPEN} {{\textsc{Open}}}

% This is for the agmin
\newcommand{\argmin}{\operatornamewithlimits{argmin}}


\newcommand{\ch}[1]
{{\color{red} #1}}

\newcommand{\qed}{\hfill\ensuremath{\blacksquare}}
\newcommand{\astar}{A$^*$}
\newcommand{\wastar}{WA$^*$}
\newcommand{\arastar}{ARA$^*$}
\newcommand{\pts}{PS}
\newcommand{\dps}{DPS}
\newcommand{\ees}{EES}
\newcommand{\gbfs}{GBFS}
\newcommand{\bees}{BEES}
\newcommand{\beeps}{BEEPS}
\newcommand{\bss}{BSS}
\newcommand{\bssb}{BSS$(B)$}
\newcommand{\bcs}{BCS}
\newcommand{\bcsb}{BCS$(C)$}
\newcommand{\ar}{AR}
\newcommand{\nr}{NR}
\newcommand{\icl}{ICL}
\newcommand{\nrr}{NRR}
\newcommand{\nrrf}{NRR1}
\newcommand{\nrrs}{NRR2}
\newcommand{\open}{\textsc{Open}}
\newcommand{\closed}{\textsc{Closed}}
\newcommand{\focal}{\textsc{Focal}}
\newcommand{\focalc}{\textsc{Focal$(C)$}}
\newcommand{\focals}{Focal Search}
\newcommand{\fmin}{$f_{min}$}


\setlength{\pdfpagewidth}{8.5in} \setlength{\pdfpageheight}{11in}
%\pdfinfo{
%/Title(To Reopen or Not to Reopen)
%/Author(Submission #)}




\setcounter{secnumdepth}{2}
\newtheorem{theorem}{Theorem}

% Used for commenting
\newcommand{\commenter}[3]{$[$\uppercase{#1}#2:#3$]$  \\}
\newcommand{\ariel}[2]{\commenter{ariel}{#1}{#2}}
\newcommand{\roni}[2]{\commenter{roni}{#1}{#2}}
\newcommand{\vitali}[2]{\commenter{vitali}{#1}{#2}}

\newcommand{\MEMO}[1]
{ \fbox{
\begin{minipage}[b]{7.9 cm}
#1
\end{minipage}
} }


\begin{document}
\title{Model-Free Planning}

\author{XXX}

\maketitle

\begin{abstract}%\vspace{-0.2cm}
TBD
\end{abstract}


\section{Introduction}
In classical planning problems, a model of the acting agent and its relation to the relevant world is given in the form of some of planning description language, e.g., the classical STRIPS model~\cite{fikes1971strips} or PDDL~\cite{mcdermott1998pddl}. This model is used to generate a plan, e.g., that achives some given goal condition. In this paper we address the problem of generating such a plan in cases where no model of the world is given. Instead, the input to the problem includes a set of observed trajectories, e.g., a set of plans that the agent has executed succesfully in the past. 
\[\ldots\]

In this paper we introduce a theoretical framework for learning a partial classical planning model from a given set of observed trajectories, and how to plan using these observed trajectoriesin a way that will ensure the goal is achieved. 



\section{Problem Definition}
The setting we address in this work is a STRIPS planning problem $\Pi=\langle P, A, I, G\rangle$, where $P$ is the set of predicates, $A$ is a set of actions, $I$ is the initial state, and $G$ is the goal condition. We assume here that states in the world as well as the goal condition are defined as a conjunction of predicates from $P$. 

In the planning problem we deal with in this paper, which we refer to as {\em model-free planning}, we do not know the capabilities of the action agent directly. Instead, we are given a set of {\em trajectories}. 
\begin{definition}[Trajectory]
A trajectory $T=\langle s_1, a_1, s_2, a_2, \ldots, a_{n-1}, s_n\rangle$ is an alternating sequence of states ($s_1,\ldots,s_n$) and actions ($a_1,\ldots,a_n$) that starts and ends with a state. 
\end{definition}
The model-free planning problem is defined as follows.
\begin{definition}[Model-free planning]
The model-free planning problem is defined by the tuple $\rangle P,I,G, \mathcal{T}\rangle$, 
where $P$, $I$, and $G$ are the predicates, initial state, and goal condition, and $\mathcal{T}$ is a set of trajectories. The task is to generate plan that will achieve $G$, i.e., a sequence of action that, if applied to $I$, and results in a state that satisfies $G$.
\end{definition}

\section{Learning a Conservative Model}
We propose a conservative approach to solve the model-free planning problem
that first learns a partial model from the given trajectories and then uses it to generate a plan that is guaranteed to reach the goal. For a real, unknown, planning problem $\Pi^*=\langle P^*,A^*,I^*,G^*\rangle$ we generate the following planning $\Pi=\langle P,A,I,G\rangle$  be the partial model 


From the set of observed trajectories ($\mathcal{T}$), we extract the set of actions that were part of the observed trajectories, denoted by $A(\mathcal{T})$. Formally,
\[ A(\mathcal{T})=\{a | \exists T\in\mathcal{T} \text{~such that~} a\in T\} \]
Next, we generate for action $a\in A(\mathcal{T})$ a set of preconditions and effects, by considering the states that were before and after $a$ in the trajectories where it appears. 
Let $T(a)$ be the set of trajectories in $\mathcal{T}$ in which $a$ appeared, i.e., 
\[ T(a)=\{T | T\in \mathcal{T} \text{~and~} a\in T\} \]


We define the preconditions and effects of $a$, denoted $pre(a)$ and $eff(a)$, respectively, as follow:
\begin{itemize}
    \item {\bf Preconditions.} The intersection of predicates that were true in all the state that preceded $a$ in the given trajectories $\mathcal{T}$. 
    \item {\bf Effects.} The difference, predicates, between states immediately before $a$ and states immediately after $a$. 
\end{itemize}

The effects are, of course, the real effects of $a$. However, the preconditions are a superset of the preconditions of $a$. Thus, every plan generated with these actions is a valid plan in the real model, but we might not find a plan we can generate even though such exists. 

MAYBE TALK ABOUT COSTS (E.G., ALL POSSIBLE SOLUTIONS TO A LINEAR EQUATION, IF WE HAVE THE COSTS OF THE TRAJECTORIES)
% hich learns from the observed trajectories a partial model that is ``safe'' to use in planning.  This results in a model that can generate a subset of the plans that the original, unknown, model can generate, but the generated plan are guaranteed to be applicable. 


%\section{Model-Free Planning as Conformant Planning}
%HERE WE'LL TALK ABOUT HOW THE ABOVE COMPILES TO CONFORMANT PLANNING,SO WE CAN JUST RUN A SUITABLE PLANNING. ALSO, SOME OF THESE PLANNERS JUST COMPILE TO CLASSICAL PLANNERS, SO THAT'S EVEN BETTER. 


\section{Thoeretical Guarantees and Limitations}



How many trajectories do we need to guarantee that every plan can be generated?


\section{Related Work}
Mour{\~{a}}o et al.~\shortcite{mourao2012learning} addressed the problem of learning a STRIPS actions given a set of observed trajectories. In their setting, the objserved trajectories consists cases in which the agent tried to perform an action an failed, while in our case the observations are only successful trajectories. Also, we assume full observability while they considered partial and noisy observations of the states in the trajectories. Their approach was to use machine learning methods to predict the outcome of an action and to predict whether it is applicable. Consequently, the plan generated by using this model can fail. By contrast, we aim for a plan that is guaranteed to work. Another similarity of this paper to the paper Mour{\~{a}}o et al.~\shortcite{mourao2012learning} 

Also related is the work of Konidarid et al.~\shortcite{konidaris2014constructing} showed how to learn a STRIPS model that provides a useful high-level plan for a continuous world. 


\newpage

\bibliography{library}
\bibliographystyle{aaai}

\end{document}
